\documentclass[12pt,letterpaper]{article}
\usepackage{fullpage}
\usepackage[top=2cm, bottom=4.5cm, left=2.5cm, right=2.5cm]{geometry}
\usepackage{amsmath,amsthm,amsfonts,amssymb,amscd}
\usepackage{lastpage}
\usepackage{enumerate}
\usepackage{fancyhdr}
\usepackage{mathrsfs}
\usepackage{xcolor}
\usepackage{graphicx}
\usepackage{listings}
\usepackage{hyperref}
\usepackage{minted}

\hypersetup{%
  colorlinks=true,
  linkcolor=blue,
  linkbordercolor={0 0 1}
}
 
\renewcommand\lstlistingname{High Performance Computing}
\renewcommand\lstlistlistingname{High Performance Computing}
\def\lstlistingautorefname{HPC}

\lstdefinestyle{Python}{
    language        = Python,
    frame           = lines, 
    basicstyle      = \footnotesize,
    keywordstyle    = \color{blue},
    stringstyle     = \color{green},
    commentstyle    = \color{red}\ttfamily
}

\setlength{\parindent}{0.0in}
\setlength{\parskip}{0.05in}

% Edit these as appropriate
\newcommand\course{High Performance Computing}
\newcommand\hwnumber{1}                  % <-- homework number
\newcommand\NetIDa{amr1215}           % <-- NetID of person #1
\newcommand\NetIDb{Agnitha Mohan Ram}           % <-- NetID of person #2 (Comment this line out for problem sets)

\pagestyle{fancyplain}
\headheight 35pt
\lhead{\NetIDa}
\lhead{\NetIDa\\\NetIDb}                 % <-- Comment this line out for problem sets (make sure you are person #1)
\chead{\textbf{Assignment 2}}
\rhead{\course \\ \today}
\lfoot{}
\cfoot{}
\rfoot{\small\thepage}
\headsep 1.5em

\begin{document}

	

{\underline{System and Compiler Specifications:}} \\

\begin{tabular}{ |c|c| } 
 \hline
 Operating System & macOS Catalina  \\
 \hline
 Processor & 1.4 GHz Quad-Core Intel Core i5 \\
 \hline
 Memory & 8 GB 2133 MHz LPDDR3  \\ 
 \hline 
Compiler version & Apple clang version 11.0.0 \\

 \hline
 Max Memory Bandwidth of processor & 37.5 GB/s \\
 \hline
 Gflops per core & 4.83	Gflops/core\\
 \hline
 Gflops per computer & 38.55 Gflops/computer\\
 \hline
\end{tabular}
\\\


\underline{\textbf{2. Matrix-matrix multiplication}} \\\\
Q - Measure performance for different loop arrangements and
try to reason why you get the best performance for a particular order?\\
A -The following order of loops gives the best performance: 
\begin{minted}{c++}
 for (long j = 0; j < n; j++) {
    for (long p = 0; p < k; p++) {
      for (long i = 0; i < m; i++) {
        double A_ip = a[i+p*m];
        double B_pj = b[p+j*k];
        double C_ij = c[i+j*m];
        C_ij = C_ij + A_ip * B_pj;
        c[i+j*m] = C_ij;
      }
    }
  }
\end{minted}
This is true because this arrangement reduces the number of cache misses since the elements of the array a stored in column major form are accessed consecutively within the inner loop. It also optimises the cache-line loading of matrix b since every element of a column in array a is multiplied with a single element from array b before proceeding to the next iteration. \\\\

 

Q - Experiment with different values for BLOCK\_SIZE (use multiples of 4) and measure performance.  What is the optimal value for BLOCK\_SIZE? \\
A - BLOCK\_SIZE = 64 is the optimal value for BLOCK\_SIZE. 
\\\\


Q - What percentage of the peak FLOP-rate do you achieve with your code? \\
95.68\%
\newpage

Blocked version:\\\\
\begin{tabular}{ |c|c|c|c|c| } 
 \hline 
Dimension   &    Time  &  Gflop/s  &     GB/s   &       Error\\
 \hline \hline
  
        64  & 0.225369 &  8.875039 &  2.218760 & 0.000000e+00 \\
       128  & 0.231605 &  8.638341 &  1.619689 & 0.000000e+00 \\
       192  & 0.214737 &  9.360847 &  1.560141 & 0.000000e+00 \\
       256  & 0.218420 &  9.217406 &  1.440220 & 0.000000e+00 \\
       320  & 0.211615 &  9.600529 &  1.440079 & 0.000000e+00 \\
       384  & 0.215512 &  9.458553 &  1.379372 & 0.000000e+00 \\
       448  & 0.222872 &  9.682551 &  1.383222 & 0.000000e+00 \\
       512  & 0.238426 &  9.006919 &  1.266598 & 0.000000e+00 \\
       576  & 0.241957 &  9.477865 &  1.316370 & 0.000000e+00 \\
       640  & 0.225641 &  9.294197 &  1.277952 & 0.000000e+00 \\
       704  & 0.223467 &  9.368193 &  1.277481 & 0.000000e+00 \\
       768  & 0.297062 &  9.149299 &  1.238968 & 0.000000e+00 \\
       832  & 0.250692 &  9.189449 &  1.237041 & 0.000000e+00 \\
       896  & 0.313063 &  9.190778 &  1.230908 & 0.000000e+00 \\
       960  & 0.385793 &  9.173168 &  1.223089 & 0.000000e+00 \\
      1024  & 0.244768 &  8.773547 &  1.165237 & 0.000000e+00 \\
      1088  & 0.285350 &  9.026904 &  1.194737 & 0.000000e+00 \\
      1152  & 0.342802 &  8.919573 &  1.176888 & 0.000000e+00 \\
      1216  & 0.395817 &  9.085237 &  1.195426 & 0.000000e+00 \\
      1280  & 0.482499 &  8.692876 &  1.140940 & 0.000000e+00 \\
      1344  & 0.545009 &  8.908901 &  1.166642 & 0.000000e+00 \\
      1408  & 0.607285 &  9.192749 &  1.201325 & 0.000000e+00 \\
      1472  & 0.694717 &  9.182174 &  1.197675 & 0.000000e+00 \\
      1536  & 0.822615 &  8.810631 &  1.147218 & 0.000000e+00 \\
      1600  & 0.896915 &  9.133530 &  1.187359 & 0.000000e+00 \\
      1664  & 1.000669 &  9.208725 &  1.195363 & 0.000000e+00 \\
      1728  & 1.237072 &  8.341924 &  1.081361 & 0.000000e+00 \\
      1792  & 1.339358 &  8.593050 &  1.112493 & 0.000000e+00 \\
      1856  & 1.515705 &  8.436246 &  1.090894 & 0.000000e+00 \\
      1920  & 1.627253 &  8.699186 &  1.123645 & 0.000000e+00 \\
      1984  & 1.674520 &  9.327487 &  1.203547 & 0.000000e+00 \\
 
\hline
\end{tabular}
\\\\
\newpage
Blocked version with OpenMP:\\\\
\begin{tabular}{ |c|c|c|c|c| } 
 \hline 
Dimension   &    Time  &  Gflop/s  &     GB/s   &       Error\\
 \hline \hline

        64  & 0.342900 &  5.833067 &  1.458267 & 0.000000e+00 \\
       128  & 0.136618 & 14.644359 &  2.745817 & 0.000000e+00 \\
       192  & 0.081910 & 24.540596 &  4.090099 & 0.000000e+00 \\
       256  & 0.065207 & 30.874997 &  4.824218 & 0.000000e+00 \\
       320  & 0.060056 & 33.828693 &  5.074304 & 0.000000e+00 \\
       384  & 0.052039 & 39.171232 &  5.712471 & 0.000000e+00 \\
       448  & 0.050796 & 42.483058 &  6.069008 & 0.000000e+00 \\
       512  & 0.051602 & 41.616287 &  5.852290 & 0.000000e+00 \\
       576  & 0.073437 & 31.227252 &  4.337118 & 0.000000e+00 \\
       640  & 0.060778 & 34.505117 &  4.744454 & 0.000000e+00 \\
       704  & 0.059087 & 35.430501 &  4.831432 & 0.000000e+00 \\
       768  & 0.070382 & 38.616535 &  5.229322 & 0.000000e+00 \\
       832  & 0.056469 & 40.796215 &  5.491798 & 0.000000e+00 \\
       896  & 0.068618 & 41.932037 &  5.615898 & 0.000000e+00 \\
       960  & 0.083264 & 42.502690 &  5.667025 & 0.000000e+00 \\
      1024  & 0.052672 & 40.770877 &  5.414882 & 0.000000e+00 \\
      1088  & 0.070169 & 36.708902 &  4.858531 & 0.000000e+00 \\
      1152  & 0.079357 & 38.530282 &  5.083857 & 0.000000e+00 \\
      1216  & 0.090087 & 39.917984 &  5.252366 & 0.000000e+00 \\
      1280  & 0.102114 & 41.074720 &  5.391057 & 0.000000e+00 \\
      1344  & 0.116308 & 41.746322 &  5.466780 & 0.000000e+00 \\
      1408  & 0.133902 & 41.691824 &  5.448363 & 0.000000e+00 \\
      1472  & 0.156692 & 40.710516 &  5.310067 & 0.000000e+00 \\
      1536  & 0.177065 & 40.932750 &  5.329785 & 0.000000e+00 \\
      1600  & 0.213586 & 38.354574 &  4.986095 & 0.000000e+00 \\
      1664  & 0.234263 & 39.335644 &  5.106069 & 0.000000e+00 \\
      1728  & 0.249849 & 41.303190 &  5.354117 & 0.000000e+00 \\
      1792  & 0.287522 & 40.028833 &  5.182304 & 0.000000e+00 \\
      1856  & 0.317323 & 40.296039 &  5.210695 & 0.000000e+00 \\
      1920  & 0.343162 & 41.251001 &  5.328254 & 0.000000e+00 \\
      1984  & 0.364777 & 42.818116 &  5.524918 & 0.000000e+00 \\
 
\hline
\end{tabular}
  

\newpage
  
  
\underline{\textbf{4. OpenMP version of 2D Jacobi/Gauss-Seidel smoothing}} \\
\begin{enumerate}

    \item Jacobi Method Performance \\
  
    \textbf{N vs Time(max iterations=1000):} \\\\
    \begin{tabular}{ |c|c| } 
 \hline
  N & Time  \\ 
 \hline \hline 
  100    &  0.074102  \\
  1000 & 6.598017  \\
  2000 & 30.472632 \\ 
\hline 
\end{tabular}
\\
 Serial with O3\\
    \textbf{N vs Time(max iterations=1000):} \\\\
    \begin{tabular}{ |c|c| } 
 \hline
  N & Time  \\ 
 \hline \hline 
  100    &  0.012708  \\
  1000 & 1.557171  \\
  2000 & 6.677376 \\ 
\hline 
\end{tabular}
  
    
    \textbf{Time taken for different N and number of threads combination(max iterations=1000):} \\\\
\begin{tabular}{ |c|c|c|c|c| } 
 \hline
  N & 1 thread & 2 threads & 4 threads & 8 threads  \\ 
 \hline \hline 
  100 & 0.086903 & 0.095865   & 0.093669 &  0.122938 \\
  1000 & 7.128139 & 4.982668 & 3.752380 & 3.788351 \\
  2000 & 29.021817 & 19.406999 & 14.457567 & 14.740195 \\
 
\hline 
\end{tabular}
\\\\
With O3 optimisation: \\\
\begin{tabular}{ |c|c|c|c|c| } 
 \hline
  N & 1 thread & 2 threads & 4 threads & 8 threads  \\ 
 \hline \hline 
  100 & 0.013756 & 0.033658  &  0.038807 & 0.049960 \\
  1000 & 1.552268 & 1.279198 & 1.071452 & 1.049622 \\
  2000 & 6.684382 & 5.228397 & 4.283326 &  4.343217 \\
 
\hline 
\end{tabular}
\\\\

\item Gauss-Seidel Method Performance \\

    \textbf{N vs Time(max iterations=1000):} \\\\
    \begin{tabular}{ |c|c| } 
 \hline
  N & Time  \\ 
 \hline \hline 
  100    & 0.041159   \\
  1000 & 3.944075  \\
  2000 & 16.062421 \\ 
\hline 
\end{tabular} \\
  Serial with O3\\
    \textbf{N vs Time(max iterations=1000):} \\\\
    \begin{tabular}{ |c|c| } 
 \hline
  N & Time  \\ 
 \hline \hline 
  100    &  0.023147  \\
  1000 &  2.574167 \\
  2000 & 10.791395 \\ 
\hline 
\end{tabular}
  

    \textbf{Time taken for different N and number of threads combination(max iterations=1000):} \\\\
\begin{tabular}{ |c|c|c|c|c| } 
 \hline
  N & 1 thread & 2 threads & 4 threads & 8 threads  \\ 
 \hline \hline 
  100 & 0.079111 & 0.071187  & 0.070732 & 0.079121 \\
  1000 & 8.829857 &  4.814917 & 2.554686 & 2.547945 \\
  2000 & 35.642225 & 19.973257 & 10.654128  & 10.711246 \\
 
\hline 
\end{tabular}
\\\\
With O3 optimisation: \\\
\begin{tabular}{ |c|c|c|c|c| } 
 \hline
  N & 1 thread & 2 threads & 4 threads & 8 threads  \\ 
 \hline \hline 
  100 & 0.026183 &  0.064224  &  0.076025 & 0.105276 \\
  1000 &  2.790062 & 1.279198 & 1.284318 & 1.339381 \\
  2000 & 10.738788 & 6.651428 & 5.377109 &  5.549504 \\
 
\hline 
\end{tabular}
\\\\




\end{enumerate}




\end{document}


